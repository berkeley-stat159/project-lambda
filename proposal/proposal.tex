\documentclass[11pt]{article}
\bibliographystyle{siam}

\title{The title of your project proposal}
\author{
  Daks, Alon\\
  \texttt{alondaks}
  \and
  Yu, Lisa Ann\\
  \texttt{lisaannyu}
  \and
  Luo, Ying\\
  \texttt{yingtluo}
  \and
  Chang, Jordeen\\
  \texttt{Jodreen}
}

\begin{document}
\maketitle

Identify a published fMRI paper and the accompanying data
\cite{lindquist2008statistical}.  You should explain the basic idea of the
paper in a paragraph.  You should also perform basic sanity check on the data
(e.g., can you downloaded, can you load the files, confirm that you have the
correct number of subjects).

Hanke, Michael, et al. "A high-resolution 7-Tesla fMRI dataset from complex 
natural stimulation with an audio movie." Scientific data 1 (2014).

Most fMRI studies use highly simplified stimulations that are vastly dissimilar
from what people experience in everyday life.  Hanke et al. sought to create a
dataset of naturally occurring brain states by exposing participants to a 
more complex stimulus, the audio description of Forrest Gump.  This particular
audio description allows for the study of auditory attention and cognition, 
language and music perception, and retrieval of explicit memory without the
effect of visual imagery.  In addition, this dataset uses inter-individual 
synchronicity to study responses to complex processing.  This paper explains
the preparation of the dataset, such as collecting physiological information
then accounting for it in the voxels, and an initial analysis.  Hanke et al.
used representational similarity anallysis to identify similar patterns
across brains.  They created a representationaly consistency map using pairwise
correlations.  After thresholding, the map revealed two clusters in the PT,
which is known to be used for auditory cognition.
could be used to study common response

Briefly explain what approach you intend to take for exploring
the data and paper.  If you intend to ``reproduce'' some aspect of the paper,
explain in exactly what sense you mean to do this.  If you are thinking about
validating the data describe the assumpution used in the analysis and indicate
how you might check them.

\bibliography{project}

\end{document}
